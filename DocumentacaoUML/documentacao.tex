\documentclass[12pt]{report}
\usepackage[a4paper,top=1.0cm,bottom=1.5cm,left=1.5cm,right=1.5cm]{geometry}
\usepackage[brazil]{babel}
\usepackage[utf8]{inputenc}    
%\usepackage[latin1]{inputenc}
\usepackage{amssymb}
\usepackage{graphicx}
\usepackage{epsfig}
\usepackage{enumerate}
\usepackage{pgf}%,pgfarrows,pgfnodes,pgfautomata,pgfheaps,pgfshade}
\usepackage{tikz-uml}

\usepackage{hyperref}

\usepackage{color}

\definecolor{pblue}{rgb}{0.13,0.13,1}
\definecolor{pgreen}{rgb}{0,0.5,0}
\definecolor{pred}{rgb}{0.9,0,0}
\definecolor{pgrey}{rgb}{0.46,0.45,0.48}


\hypersetup{
    bookmarks=true,         % show bookmarks bar?
    unicode=false,          % non-Latin characters in Acrobat’s bookmarks
    pdftoolbar=true,        % show Acrobat’s toolbar?
    pdfmenubar=true,        % show Acrobat’s menu?
    pdffitwindow=false,     % window fit to page when opened
    pdfstartview={FitH},    % fits the width of the page to the window
    pdftitle={My title},    % title
    pdfauthor={Author},     % author
    pdfsubject={Subject},   % subject of the document
    pdfcreator={Creator},   % creator of the document
    pdfproducer={Producer}, % producer of the document
    pdfkeywords={keyword1, key2, key3}, % list of keywords
    pdfnewwindow=true,      % links in new PDF window
    colorlinks=false,       % false: boxed links; true: colored links
    linkcolor=red,          % color of internal links (change box color with linkbordercolor)
    citecolor=green,        % color of links to bibliography
    filecolor=magenta,      % color of file links
    urlcolor=blue           % color of external links
}

\usepackage{listings}
%\lstset{language=Java,
%  showspaces=false,
%  showtabs=false,
%  breaklines=true,
%  showstringspaces=false,
%  breakatwhitespace=true,
%  commentstyle=\color{pgreen},
%  keywordstyle=\color{pblue},
%  stringstyle=\color{pred},
%  basicstyle=\ttfamily,
%  moredelim=[il][\textcolor{pgrey}]{$$},
%  moredelim=[is][\textcolor{pgrey}]{\%\%}{\%\%}
%}

\lstset{ %
  backgroundcolor=\color{white}, 
  basicstyle=\footnotesize,       
  breakatwhitespace=false,        
  breaklines=true,                 
  captionpos=b,                    
  commentstyle=\color{pgreen},   
  escapeinside={\%*}{*)},        
  extendedchars=true,              
  frame=single,                  
  keywordstyle=\color{blue},       
  language=Java,                
  numbers=left,                    
  numbersep=5pt,                   
  numberstyle=\tiny\color{pgrey},
  rulecolor=\color{black},        
  showspaces=false,               
  showstringspaces=false,          
  showtabs=false,                  
  stepnumber=2,                    
  stringstyle=\color{pred},   
  tabsize=2,                      
  title=\lstname, 
  xleftmargin=\parindent,
%  xleftmargin=.25in,
%  xrightmargin=.25in,
  morekeywords={not,\},\{,preconditions,effects },            
  deletekeywords={time}            
}

%%%%%%%%%%%%%%%%%%%%%%%%%%%%%%%%%%%%%%%

\usepackage{fancyhdr}

\fancypagestyle{primeira}{
\fancyhead{}
\fancyfoot{}
\lhead{\theauthor}
\rhead{Bons Estudos! \thetitle}
\rfoot{\thepage}
}


%%%%%%%%%%%%%%%%%%%%%%%%%%%%%%%%%%%%%%%%
\begin{document}\pagestyle{empty}
\begin{titlepage}
  \vfill
    \begin{center}
      {\large \textbf{UNIVERSIDADE FEDERAL DOS VALES DO JEQUITINHONHA E MUCURI}} \\[0.5cm]
      {\large \textbf{SISTEMAS DE INFORMAÇÃO}} \\[2cm]

      {\large \textbf{Gabriel Morais}}\\[0.4cm]
      {\large \textbf{Tales Félix}}\\[0.4cm]
      {\large \textbf{Ulisses Brandão}}\\[4cm]
  
      {\Large Análise Orientada a Objetos e Uml}\\[4cm]
  
      \hspace{.45\textwidth} %posiciona a minipage
      \begin{minipage}{.5\textwidth}
      \large
      Docente: Eduardo Pelli.\\[8cm]
  \end{minipage}
  {\large Diamantina, 18 de Outubro de 2020}
  \end{center}
\end{titlepage}

\newpage
\section{Introdução}
Uma loja de materiais de construção precisa de um sistema para gerenciar
o negócio. A loja conta com um galpão para estoque de materiais, e atendimento
presencial a clientes. O sistema deve possibilitar o cadastro de materiais,
com preço, quantidade em estoque, especificação do produto, margem de lucro,
data de fabricação, fornecedor e outras informações que considerarem relevantes.
Além disso, o sistema deverá fazer a gestão do relacionamento com o cliente,
guardando informações pessoais (Nome, endereço, número do telefone, e-mail e CPF)
e registro de compras. O sistema deverá ser utilizado pelos colaboradores da
loja durante o atendimento ao cliente para gerar o extrato de venda e associar
essas informações aos clientes da loja. O administrador poderá cadastrar
clientes, colaboradores, produtos (materiais) e consultar as vendas.



\section{Diagrama de Casos de Uso}

\begin{tikzpicture}

  \begin{umlsystem}{Casos de uso}
   %Casos de uso
  \umlusecase[x=-5,y=10]{Realizar Venda}
  \umlusecase[x=1,y=8]{Gerar Extrato de Vendas}
  %\umlusecase[x=-8,y=9]{Consultar Estoque}
  \umlusecase[x=-5,y=8]{cadastrar Colaborador}
  \umlusecase[x=-5,y=6]{Cadastrar cliente}
  %\umlusecase[x=1,y=5]{Gerenciar Cliente}
  \umlusecase[x=-5,y=4]{cadastrar materiais}
  \umlusecase[x=-5,y=2]{Consultar Vendas}
  \umlusecase[x=1,y=1]{Consultar Vendas por cliente}


  
  \end{umlsystem}
  
  
  % \umlextend  {usecase-3}{usecase-6}
  
  
  %atores
  \umlactor[x=-12,y=10]{Colaborador}
  \umlactor[x=-12,y=5]{Administrador}
  
  %include
  %\umlinclude{usecase-1}{usecase-2}
  
  %extend
  \umlextend{usecase-6}{usecase-7}
  %\umlextend{usecase-5}{usecase-4}

  %Ator relacionamentos
  \umlassoc{Colaborador}{usecase-1}
  \umlassoc{Colaborador}{usecase-2}
  \umlassoc{Administrador}{usecase-3}
  \umlassoc{Administrador}{usecase-4}
  \umlassoc{Administrador}{usecase-5}
  \umlassoc{Administrador}{usecase-6}
  %\umlassoc{Administrador}{usecase-8}
  \umlinherit{Administrador}{Colaborador}
  
  
\end{tikzpicture}\\

  O diagrama de casos de uso corresponde a uma visão externa do sistema
  e representa graficamente os atores, os casos de uso, e os relacionamentos
  entre estes elementos. Ele tem como objetivo ilustrar em um nível alto de
  abstração quais elementos externos interagem com que funcionalidades do sistema,
  ou seja, a finalidade de um diagrama de caso de uso é apresentar um tipo de diagrama
  de contexto que apresenta os elementos externos de um sistema e as maneiras segundo
  as quais eles as utilizam.

\section{Fluxo de Eventos}
\subsection{Realizar Venda}
- O Caso de uso começa quando o Colaborador/Administrador acessa a opção “Realizar venda”;\\
- O Colaborador/Administrador deverá estar logado na aplicação;\\
- Caso o cliente não esteja cadastrado, o colaborador deverá acessar a opção
“Solicitar cadastro” para assim o Administrador realizar o cadastro;\\
- Com o Cliente cadastrado, o Colaborador/Administrador entrar com as
informações de compras nos campos que aparece no sistema;\\
- Colaborador/Administrador acessa a opção “Gerar Extrato” para gerar um
extrato da venda feita para o cliente;\\
- Sair do Sistema ou escolher outra opção;\\

\subsection{Gerar Extrato de Vendas}
- O Caso de uso começa quando o Colaborador/Administrador acessa a opção “Gerar Extratos de vendas”;\\
- O Colaborador/Administrador deverá estar logado na aplicação;\\
- Com o Cliente cadastrado, o Colaborador/Administrador acessa a opção “Gerar Extrato”;\\
- Sair do Sistema ou escolher outra opção;\\

\subsection{Cadastrar Colaborador}
- O caso de uso começa quando o Administrador precisa cadastrar um novo colaborador.\\
- Com o login efetuado o Administrador deverá acessar o “Cadastro Colaborador”;\\
- Através de um formulário o administrador deverá informar os seguintes dados:
nome, CPF, email e senha;\\
- Sair do sistema ou escolher outra opção;\\

\subsection{Cadastrar Cliente}
- O caso de uso começa quando o Administrador precisa efetuar o cadastro de um novo cliente;\\
- Com o login efetuado o Administrador deverá acessar o “Cadastro de Cliente”;\\
- Através de um formulário o administrador deverá informar os seguintes dados:\\
nome, endereço, número do telefone, e-mail e CPF;\\
- Sair do sistema ou escolher outra opção;\\

\subsection{Gerenciar Cliente}
- O caso de uso começa quando o Administrador/Colaborador precisa efetuar uma
busca nos dados do cliente;\\
- Com o login efetuado o Administrador/Colaborador deverá acessar o “Gereciar
Cliente” e fazer a pesquisa entre os clientes cadastrados;\\
- Sair do sistema ou escolher outra opção;\\

\subsection{Cadastrar Materiais}
- O caso de uso começa quando o Administrador precisa cadastrar novos
produtos(materiais) no sistema;\\
- Com o login efetuado o Administrador deverá acessar “Cadastrar Materiais”
informando dados do material.  com preço, quantidade em estoque, especificação
do produto, margem de lucro, data de fabricação e fornecedor;\\
- Sair do sistema ou escolher outra opção.\\

\subsection{Consultar Estoque}
- O caso de uso começa quando o Administrador/Colaborador precisa realizar uma
consulta de produtos(materiais) no sistema;\\
- Com o login efetuado o Administrador deverá acessar “Consultar Estoque”
informando dados para a pesquisa do produto;\\
- Sair do sistema ou escolher outra opção;\\

\subsection{Consultar Vendas}
- Com o login efetuado o Administrador deverá acessar “Consultar Vendas” escolhendo
a opção “consultar todas as vendas”, e esperar uma mensagem mostrando o relatório das vendas;\\
- Sair do sistema ou escolher outra opção.\\


\subsection{Consultar Vendas por Cliente}
- Com o login efetuado o Administrador deverá acessar “Consultar Vendas”
escolhendo a opção de consultar “Consultar por cliente”, e esperar uma mensagem 
mostrando o relatório das vendas;\\
- Sair do sistema ou escolher outra opção.\\


\section{Classes}
{
\scalebox{.67}
{
\begin{tikzpicture}
   %SISTEMA
   \umlclass[x = , y =0]{Sistema} {
   - colaboradores: Colaborador[25]\\
   - vendas: Venda[ ]\\
   - clientes: Cliente[ ]\\
   - estoque: Material[ ]\\
   
    }{
   + incluirColaborador(colaborador: Colaborador): bool\\
   + incluirCliente(cliente: Cliente): bool\\
   + realizarVenda(venda: Venda): bool\\
   + cancelarVenda(venda: Venda):bool\\
    + incluirMaterial(material: Material) : bool \\
   + ConsultarVendas() : bool
  
   }
   
    %PESSOA
    \umlclass[x=10, y=-8]{Pessoa}{
    - nome : String \\
    - endereco : String \\

    - email : String \\
    - cpf : String \\
    - telefone : String \\}{
    }
    
    
    %Cliente
    \umlclass[x=3, y=-15]{Cliente}{
    - regCompras : Venda[ ]}{
    + ImprimirInfoCliente() : bool}
    
    %Colaborador
    \umlclass[x=11, y=-15]{Colaborador}{
    - login : String \\
    - senha : String \\}{}
    
     %ADM
    \umlclass[x=10, y=-20]{Adm}{}{}
  
   
    %Venda
    \umlclass[x =0, y=-8]{Venda}{
     - idVenda: int \\
     - data : Date \\
    - valorTotal: float \\
    - materiais: Material[ ] \\ }{
      + GerarExtrato() : bool
      }
     


   %Material
   \umlclass[x=-10, y=-7]{Material} {
   - nome: String \\
   - quantidade: int \\
   - preco: float \\
   - especificacao: String \\
   - margemLucro: float \\
   - dataFabricacao: Date \\
   - fornecedor: String \\
   } {
  
   }
   \umlclass[x=-10, y=-12]{Arquivo}{}{}
\end{tikzpicture}
}
}



\section{Diagrama de Sequência}

\subsection{Diagrama de Sequência Cadrastar Colaborador}
\begin{tikzpicture} 
  \begin{umlseqdiag} 
    \umlactor[]{ADM}
    \umlobject[stereo=entity]{Login}
    \umlobject[stereo=entity]{Validar}
    \umlobject[stereo=entity]{TelaSistema}
    \umlobject[stereo=entity]{ValidarCodigo}
    \begin{umlcall}[op={1},return=4]{ADM}{Login}
    \end{umlcall}
    \begin{umlcall}[op={2}, return=3]{Login}{Validar}
    \end{umlcall}
    \begin{umlcall}[op={5}, dt=10]{ADM}{TelaSistema}
    \end{umlcall}
    \begin{umlcall}[op={6}, return=8]{TelaSistema}{ValidarCodigo}
    \end{umlcall}
    \begin{umlcall}[op={9}]{TelaSistema}{ADM}
    \end{umlcall}
    \begin{umlcallself}[op={7}]{ValidarCodigo} 
    \end{umlcallself} 
  \end{umlseqdiag} 
\end{tikzpicture}\\

{\bf Descrição:}\\
1. Logar()\\
2. Validar Login\\
3. Resposta validação\\
4. Login Efetuado\\
5. Formulário\\
6. CadrastarColaborador()\\
7. Validar Dados Inseridos\\
8. Mostrar mensagem de sucesso ou mostrar erro\\
9. Retonar Início\\


No que se refere ao diagrama de seqüência, preocupa-se com a ordem temporal
em que as mensagens são trocadas entre os objetos envolvidos em determinado
processo, ou seja, quais condições devem ser satisfeitas e quais métodos devem
ser disparados entre os objetos envolvidos e em que ordem durante um processo.
Dessa forma, determinar a ordem em que os eventos ocorrem, as mensagens que são
enviadas, os métodos que são chamados e como os objetos interagem entre si dentro
de um determinado processo é o principal objetivo deste diagrama.



\vspace{3 cm}



%\include{class} % DIAGRAMA
\centering
{
\scalebox{.67}
{
\begin{tikzpicture}
   %SISTEMA
   \umlclass[x = , y =0]{Sistema} {
   - colaboradores: Colaborador[25]\\
   - vendas: Venda[ ]\\
   - clientes: Cliente[ ]\\
   - estoque: Material[ ]\\
   
    }{
   + incluirColaborador(colaborador: Colaborador): bool\\
   + incluirCliente(cliente: Cliente): bool\\
   + realizarVenda(venda: Venda): bool\\
   + cancelarVenda(venda: Venda):bool\\
    + incluirMaterial(material: Material) : bool \\
   + ConsultarVendas() : bool
  
   }
   
    %PESSOA
    \umlclass[x=10, y=-8]{Pessoa}{
    - nome : String \\
    - endereco : String \\

    - email : String \\
    - cpf : String \\
    - telefone : String \\}{
    }
    
    
    %Cliente
    \umlclass[x=3, y=-15]{Cliente}{
    - regCompras : Venda[ ]}{
    + ImprimirInfoCliente() : bool}
    
    %Colaborador
    \umlclass[x=11, y=-15]{Colaborador}{
    - login : String \\
    - senha : String \\}{}
    
     %ADM
    \umlclass[x=10, y=-20]{Adm}{}{}
  
   
    %Venda
    \umlclass[x =0, y=-8]{Venda}{
     - idVenda: int \\
     - data : Date \\
    - valorTotal: float \\
    - materiais: Material[ ] \\ }{
      + GerarExtrato() : bool
      }
     


   %Material
   \umlclass[x=-10, y=-7]{Material} {
   - nome: String \\
   - quantidade: int \\
   - preco: float \\
   - especificacao: String \\
   - margemLucro: float \\
   - dataFabricacao: Date \\
   - fornecedor: String \\
   } {
     }
     


    
   %RELACIONAMENTOS
 
 
    \umlcompo[]{Sistema}{Pessoa}
   
    \umlcompo[]{Sistema}{Venda}
    \umlcompo[]{Sistema}{Material}
   
     \umlcompo[]{Cliente}{Venda}
     \umlcompo[]{Venda}{Material}
    \umlinherit[]{Cliente}{Pessoa}
    \umlinherit[]{Adm}{Colaborador}
    \umlinherit[]{Colaborador}{Pessoa}
    %\umlVHassoc{Quarto}{Venda}
   % \umlVHassoc{Colaborador}{Venda}
   % \umlVHassoc{Material}{Venda}
\end{tikzpicture}
}
}


\newpage
O diagrama de classes é considerado por muitos autores como o mais importante e
  o mais utilizado diagrama da UML. Seu principal enfoque está em permitir a visualização
  das classes que irão compor o sistema com seus respectivos atributos e métodos,
  bem como em demonstrar como as classes do sistema se relacionam, se complementam e
  transmitem informações entre si. Este diagrama apresenta uma visão estática de como
  as classes estão organizadas, preocupando-se em definir a estrutura lógica das mesmas.

  \section{Conclusão}
  A maior dificuldade estava em fazer um Diagrama e só na implementação do próximo
  perceber que faltava uma funcionalidade ou classe no sistema. Isso poderia desestruturar
  o sistema mudando o pensamento lógico de algumas partes.\\
  Além disso, tivemos dificuldade na implementação do  Diagrama de Sequência,
  uma vez que não encontramos artigos relacionados que abrangia o tema.
  Desse modo, percebemos a complexidade de abstrair um problema real,
  e implementação de tal projeto.

  \newpage
  \begin{thebibliography}{4}
    \bibitem{DEVMEDIA}DEVMEDIA.\textbf{O Que é UML e Diagramas de Caso de Uso}: Introdução Prática à UML. 
    Recuperado em 17 de outubro de 2020,
    https://www.devmedia.com.br/o-que-e-uml-e-diagramas-de-caso-de-uso-introducao-pratica-a-uml/23408

    \bibitem{DEVMEDIA}DEVMEDIA.\textbf{Artigo SQL Magazine 64 - Utilizando UML}. 
    Recuperado em 17 de outubro de 2020,
    https://www.devmedia.com.br/artigo-sql-magazine-64-utilizando-uml/12665
  
    \bibitem{DEVMEDIA}DEVMEDIA.\textbf{Diagrama de Classes UML}. 
    Recuperado em 17 de outubro de 2020,
    https://www.devmedia.com.br/diagrama-de-classes-uml/12251
    \end{thebibliography}

\end{document}